\cvsection{Experience}

\begin{cventries}

  \cventry
  {Head of AI Engineering, Founding Research Scientist}
  {Harvey AI}
  {San Francisco, CA}
  {Feb 2023 - Present}
  {
    \begin{cvitems}
      \item Joined as the first Machine Learning hire, built out numerous from-scratch product lines, and scaled and managed the AI Engineering team
      \item Developed and shipped the core Assistant AI enterprise product offering for Harvey, which is responsible for \$20M+ ARR.
      \item Scaled the AI org from zero to 10+ engineers in under a year, working with executives to set engineering culture, and scale processes.
      \item Developed and iterated on many internal abstractions and prototyping systems which power the day-to-day research
      and development of the current product lines.
      \item Worked closely with internal domain experts and external design partners (including onsite visits) to design and
      improve new product lines.
    \end{cvitems}
  }

	\cventry
	{Machine Learning Scientist \rightarrow \; Senior Machine Learning Scientist}
	{Tesla Autopilot (Computer Vision Team)}
	{Palo Alto, CA}
	{Sept 2019 - Sept 2022}
	{
		\begin{cvitems}
			\item Led a four person dataset management team that handled Autolabeler operation to create datasets.
			\begin{itemize}
				\item Designed a dataset management system for easy reproducibility and delivery of new incremental datasets.
				\item Maintained the ingestion pipeline to take in customer data and run Autolabeler Neural Networks, build new targeted datasets
        			based off customer issues, and deliver repeated wins to the in-car driving experience.
				\item Wrote a particularly lightweight Autolabeling system that addressed critical in-car issues in a regulatory context.
			\end{itemize}
			\item Trained and shipped a variety of Neural Networks for Full Self Driving. This included dataset and highloss analysis to find issues in
      			training data, optimizing dataloading code, architecture search, tuning training regime, minimizing quantization error,
      			designing task appropriate metrics, reducing in-car inference latency, and more. Some of the specific Neural Networks I worked on are listed below:
			\item Trained and shipped iterations of Emergency Vehicle Detection Neural Network (from scratch bringup)
			\begin{itemize}
				\item Worked with data labeling team to grow the dataset in targeted directions, as well as fix specific issues in labeling ontology.
				\item Extensive architecture search to profile latency performance across backbones and video modules to maximize metrics and minimize inference latency
			\end{itemize}
			\item Trained and shipped iterations of the Driver Monitoring Neural Network
			\begin{itemize}
				\item Increased accuracy significantly via high loss analysis, architecture search, and training regime tuning.
				\item Added new Neural Network outputs to address customer issues, such as when the driver was partially out of camera view.
			\end{itemize}
			\item Trained and shipped iterations of the Camera Occlusion Neural Network (patent pending, from scratch bringup)
			\begin{itemize}
				\item Added new Neural Network outputs to address customer issues, such as specific cases of difficult to detect tire spray occluding the leading vehicle.
			\end{itemize}
			\item Trained and shipped iterations of the Pedestrian Semantic Classification Neural Network (from scratch bringup)
			\begin{itemize}
				\item Increased accuracy via high loss analysis and extensive architecture and training regime tuning.
			\end{itemize}
			\item Wrote an architecture visualization graph tracer to represent the Neural Network visually along with per-layer latency, which is used across the Vision team to optimize inference latency.
		\end{cvitems}
	}

	\cventry
	{Engineer in Deep Learning Research}
	{DeepScale (acquired by Tesla)}
	{Mountain View, CA}
	{June 2018 - Sept 2019}
	{
		\begin{cvitems}
			\item Wrote the internal repository for PyTorch training used for deep learning experiments.
			\begin{itemize}
				\item Designed a generic training loop to work with arbitrary dataloaders, architectures, and metrics.
        % \item Focused on modularity via decoupled components and reproducibility via a config file to define the entire experiment.
			\end{itemize}
			\item Used PyTorch to achieve 77\% class IOU on Cityscapes dataset using a DeepLabV3+ style net.
			\item Owned freespace and lane semantic segmentation Neural Networks
			\begin{itemize}
				% \item Task definition based on discussions with business development for customer needs.
				\item Wrote annotation guide for both tasks, and QA'ed annotated data from Samasource.
				\item Redesigned the previously mentioned Cityscapes model to use 1/100th - 1/200th of the FLOPS.
        % \item Identified failure modes, tuned annotation guide and dataset to remove them.
        % \item Achieved 98\% IOU on Freespace task with 8 gigaflop model.
        % \item Achieved 78\% mIOU on Lane segmentation task with 8 gigaflop model.
				\item These two networks (along with an object detector) were displayed at CES 2019 as a flagship demo.
			\end{itemize}
			% \item Implemented multi task learning enabling the unification of these two Neural Networks (along with pixel wise depth), with negligible drop in metrics (\textasciitilde 1\%), into a single 8 gigaflop net.
			% \item Gave multiple company wide presentations on PyTorch, segmentation, and metrics best practices and techniques.
		\end{cvitems}
	}

% 	\cventry
% 	{Platform Intern}
% 	{UpGuard}
% 	{Mountain View, CA}
% 	{Summer 2017}
% 	{
% 		\begin{cvitems}
% 			\item {Worked as a Junior SRE using Kubernetes and Docker to support a front facing cybersecurity application.}
% 		\end{cvitems}
% 	}

\end{cventries}
